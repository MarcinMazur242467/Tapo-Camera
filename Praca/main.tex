\documentclass[a4paper,12pt,twoside]{article}
\usepackage[top=2.5cm,bottom=2.5cm,inner=3cm,outer=2cm, headheight=1.25cm, footskip=1.25cm]{geometry}
\usepackage{setspace}
\usepackage{fancyhdr}
\usepackage{titlesec}
\usepackage{enumitem}
\usepackage{hyperref}
\usepackage[polish]{babel}
\usepackage{pdfpages} % Dodaj pakiet pdfpages
\usepackage{graphicx}
\usepackage{float}
\usepackage[normalem]{ulem}
\usepackage[nottoc]{tocbibind}

%\hypersetup{colorlinks=true}

\usepackage[T1]{fontenc} % Użyj kodowania czcionek T1
\usepackage[utf8]{inputenc} % Użyj kodowania wejściowego UTF-8 dla znaków specjalnych
\usepackage{helvet} % Czcionka Helvetica
\renewcommand{\familydefault}{\sfdefault} % Ustaw domyślną czcionkę na bezszeryfową
\usepackage{amsmath} % Potrzebne do korzystania z środowisk matematycznych
\usepackage{chngcntr}
\counterwithin{figure}{section} % Numeracja rysunków zgodnie z numerem rozdziału
\counterwithin{table}{section}
\usepackage{amsmath}
\usepackage{tocloft}
\usepackage{nomencl}
\makenomenclature
\newcommand{\listequationsname}{\hspace{-1.0cm}\Large{Spis równań}}
\newlistof{myequations}{equ}{\listequationsname}
\newcommand{\myequations}[1]{%
    \addcontentsline{equ}{myequations}{\protect\numberline{\theequation}#1}\par
}

\usepackage{caption}
\captionsetup[figure]{name=Rys.}

\usepackage{listings}
\lstset{
    language=Python, % Wybierz język programowania
    basicstyle=\ttfamily, % Styl podstawowy
    numbers=left, % Numery linii po lewej stronie
    numberstyle=\tiny\color{gray}, % Styl numerów linii
    frame=single, % Rama wokół kodu
    breaklines=true, % Automatyczne łamanie linii
    captionpos=b, % Pozycja podpisu
    showstringspaces=false, % Nie pokazuj spacji w napisach
}

\usepackage{lmodern}
\renewcommand*\familydefault{\sfdefault}

\pagestyle{fancy}
\fancyhead{}
\fancyhead[LO,RE]{}
\fancyfoot{}
\fancyfoot[LE,RO]{\thepage}
\fancyfoot[LO,RE]{}
\renewcommand{\headrulewidth}{0pt}

\setlength{\parindent}{0.5cm}
\setlist[itemize]{label=•,itemsep=0pt}

\titleformat{\section}{\normalfont\fontsize{16}{18}\bfseries}{\thesection}{1em}{}
\titleformat{\subsection}{\normalfont\fontsize{14}{16}\bfseries}{\thesubsection}{1em}{}
\titleformat{\subsubsection}{\normalfont\fontsize{13}{15}\bfseries}{\thesubsubsection}{1em}{}

\newcommand{\MMp}[1]{\marginpar{\textcolor{blue}{\textbf{MM}: \footnotesize #1}}}
\newcommand{\MM}[1]{\textcolor{blue}{\textbf{MM}: #1}}

\usepackage[backend=bibtex,style=authoryear,natbib]{biblatex}
\addbibresource{literatura.bib}

\fancypagestyle{plain}{
    \fancyhf{} % Usuń wszystko z nagłówka i stopki
    \fancyfoot[LE,RO]{\thepage} % Dodaj numer strony na zewnętrznych krawędziach stopki
    \renewcommand{\headrulewidth}{0pt} % Brak linii w nagłówku
    \renewcommand{\footrulewidth}{0pt} % Brak linii w stopce
}

\fancypagestyle{empty}{
    \fancyhf{} % Usuń wszystko z nagłówka i stopki
    \renewcommand{\headrulewidth}{0pt} % Brak linii w nagłówku
    \renewcommand{\footrulewidth}{0pt} % Brak linii w stopce
}



\begin{document}
\counterwithin{lstlisting}{section}
\thispagestyle{empty}

\includegraphics[width=5in]{LogoPL-FTIMS.pdf}

\begin{centering}

\vspace{1cm}

\textbf{\textit{\large{Imię i Nazwisko}}}

\vspace{.1cm}

\textbf{\textit{\large{Nr albumu}}}

\vspace{1.5cm}

{\large{PRACA DYPLOMOWA}}

\vspace{.1cm}

{\large{inżynierska}}

\vspace{.1cm}

na kierunku Informatyka Stosowana

\vspace{1.8cm}

\textbf{\Large{Temat pracy w języku prowadzenia studiów Temat pracy w języku prowadzenia studiów Temat pracy w języku prowadzenia studiów}}

\end{centering}

\vspace{2.25cm}
\begin{centering}
{Instytut Informatyki I72}\\
\end{centering}
\vspace{.25cm}
\textbf{Promotor:} \dotfill

\begin{centering}
(tytuł/stopień naukowy, imię i nazwisko)

~\\

\end{centering}
\textbf{Opiekun pomocniczy:*)} \dotfill

\begin{centering}
(tytuł/stopień naukowy, imię i nazwisko)

~\\

\end{centering}
\textbf{Promotor uczelni partnerskiej:**)} \dotfill

\begin{centering}
(tytuł/stopień naukowy, imię i nazwisko)

~\\

\end{centering}

\vfill

\begin{centering}
ŁÓDŹ <tylko rok>

\end{centering}
\hspace{0.5cm}
\begin{itemize}
\small
\item[*]	jeśli został powołany
\item[**]	w przypadku procedury uznania
\end{itemize}
 % Wypełnij w tym pliku dane strony tytułowej
%\pagenumbering{roman}
%\setcounter{page}{2}
\newpage
\thispagestyle{empty}
\mbox{}
\newpage
\onehalfspacing

% Spis treści
\pagestyle{empty}
\tableofcontents
\newpage
\clearpage

\pagestyle{fancy}
\pagenumbering{arabic}
\setcounter{page}{3}

% Rozdziały pracy
\newpage
\section*{Streszczenie}
%\addcontentsline{toc}{chapter}{Streszczenie}
\label{chap:Streszczenie}
%\addcontentsline{toc}{chapter}{Streszczenie}
%\markboth{STRESZCZENIE}{STRESZCZENIE}

Lorem ipsum dolor sit amet, consectetur adipiscing elit. Sed viverra vitae justo non interdum. Nullam aliquam felis vel tellus efficitur, eget ultrices elit varius. Maecenas euismod ante sed tristique. Suspendisse vehicula efficitur ante, ac condimentum dui congue ut. Fusce in velit eu erat elementum dignissim eget non tortor. Proin sit amet est sit amet ex vehicula laoreet. Vivamus ut leo vel nisi volutpat facilisis. Praesent feugiat risus eu est gravida, id convallis nunc scelerisque. Nullam in lorem dui. Sed sit amet bibendum mauris, vel hendrerit est. Integer cursus odio eget metus sagittis, eu mattis ante vulputate. Sed hen-drerit, erat a aliquet bibendum, quam libero ultricies tortor, in sagittis ligula elit sed libero.
Pellentesque id hendrerit est. Sed gravida eros in justo accumsan, non sceler-isque dolor tincidunt. Vestibulum id nunc euismod, fermentum odio nec, tincid-unt arcu. Nunc eget leo nec eros euismod scelerisque. Nunc facilisis purus id neque laoreet interdum. Duis lacinia vestibulum mi, ut viverra est volutpat eget. Nullam aliquam nibh eu dui elementum scelerisque. Fusce efficitur libero a tel-lus convallis, in venenatis libero pellentesque. Suspendisse potenti. Fusce lacin-ia sapien ac odio tincidunt, nec feugiat tellus bibendum. Proin aliquam ex eget libero fermentum, in bibendum urna ultricies. Vivamus ut pellentesque libero. Aenean tincidunt dolor et quam hendrerit, vel ultricies ligula faucibus. Vivamus a elit eu dui bibendum fermentum id id neque. Nulla eget tortor euismod, tristique neque in, congue arcu.
Praesent tristique eget quam non dignissim. Integer placerat enim et erat con-sectetur, ut scelerisque nulla interdum. Pellentesque habitant morbi tristique senectus et netus et malesuada fames ac turpis egestas. Sed at lectus non orci dictum consectetur. Sed sagittis, eros at eleifend dignissim, metus arcu sollici-tudin libero, non egestas justo libero vel mi. Sed a urna arcu. Fusce sodales eros vel nulla fringilla posuere. 

\bigskip

\textbf{Słowa kluczowe}: \quad nie \quad więcej\quad niż \quad pięć

\textbf{Keywords}: \quad not \quad more \quad than \quad five


\newpage
\newpage
\phantomsection
\section*{Wstęp}
\addcontentsline{toc}{section}{Wstęp}

Globalny rynek systemów monitoringu przechodzi dynamiczną transformację, będącą efektem rozwoju \textbf{Internetu Rzeczy (IoT)}. Kamery IP stały się wszechobecnym elementem infrastruktury cyfrowej, pełniąc funkcje od podstawowego dozoru, aż po zaawansowaną analizę danych. Równolegle z postępem technologicznym, pojawia się wyzwanie o charakterze inżynierskim, jakim jest dominacja systemów opartych na \textbf{zamkniętym oprogramowaniu (proprietary software)}.\\



Wybór tematu pracy wynika z konieczności zaadresowania problemu \textbf{vendor lock-in} w kontekście popularnych kamer konsumenckich, na przykładzie urządzeń TP-Link Tapo. Zjawisko to, polegające na uzależnieniu pełnej funkcjonalności sprzętu od infrastruktury chmurowej i aplikacji mobilnej producenta, ogranicza \textbf{dostepność danych} oraz \textbf{możliwości integracji} z otwartymi systemami automatyki i bezpieczeństwa. Problem ten jest szczególnie istotny w kontekście \textbf{cyber bezpieczeństwa}, gdzie zamknięte i często nieaudytowane firmware może stanowić potencjalny wektor ataku.\\


W pracy zastosowano \textbf{metodykę Double Diamond}, dzieląc proces projektowy na fazy eksploracji i definiowania problemu (analiza protokołów kamery) oraz fazy rozwoju i dostarczania rozwiązania. Warstwa aplikacyjna została zaimplementowana w języku \textbf{Python 3.13} z wykorzystaniem \textbf{konteneryzacji Docker} dla zapewnienia izolacji i wysokiej \textbf{reprodukowalności środowiska}. Komunikacja z kamerą odbywa się poprzez bibliotekę \textbf{PyTapo}, natomiast przetwarzanie strumienia wideo RTSP realizują narzędzia \textbf{FFmpeg} i \textbf{OpenCV}. Taki zestaw narzędzi, osadzony w architekturze serwera \textbf{Flask} z protokołem \textbf{WebSocket's}, pozwolił na stworzenie systemu o niskim opóźnieniu (\textit{low latency}).\\

Niniejsza praca ma za zadanie stanowić nie tylko dowód kompetencji inżynierskich, ale także praktyczny wkład w rozwój otwartych technologii w dziedzinie monitoringu IoT.

\newpage
\section*{Cel i zakres pracy}
\addcontentsline{toc}{section}{Cel i zakres pracy}\label{chap:Cel_i_zakres_pracy}

\newpage
\section{Wprowadzenie technologiczne Kamer IP}
\label{sec:wprowadzenie_kamer_ip}% BRAK PUSTYCH LINII TUTAJ
Rozdział ten ma za zadanie ugruntować zrozumienie złożoności systemów kamer IP i precyzyjnie wskazać na luki w otwartych standardach, które musi wypełnić zaprojektowane rozwiązanie. \ 

Współczesne systemy monitoringu wizyjnego oparte na kamerach IP stanowią kluczowy element infrastruktury bezpieczeństwa, wykraczając funkcjonalnością poza tradycyjne, analogowe systemy CCTV. Ewolucja ta jest ściśle związana z rozwojem sieci komputerowych i koncepcji IoT, gdzie urządzenia peryferyjne uzyskują zdolność do przetwarzania i autonomicznej komunikacji w ramach sieci. Z inżynierskiego punktu widzenia, kamera IP jest zaawansowanym systemem wbudowanym, łączącym optykę, cyfrowe przetwarzanie sygnału, kompresję danych oraz kompleksowy stos protokołów sieciowych.
\subsection{Zastosowanie Kamer IP}\label{subsec:zastosowanie_kamer_ip}
Na podstawie raportu Hanwha Vision z 2025 roku, można wyróżnić następujące, główne obszary zastosowań kamer IP \cite{Hanwha:2025}:
\begin{longtable}{|p{0.25\linewidth}|p{0.71\linewidth}|} 
    \caption{Główne obszary zastosowań kamer IP (na podstawie raportu Hanwha Vision, 2025)}
    \label{tab:zastosowanie_kamer_ip} \\ % Komenda caption musi być na początku longtable

    \hline
    \textbf{Obszar zastosowania} & \textbf{Przykłady wykorzystania kamer IP} \\
    \hline
    \endhead % Koniec nagłówka tabeli, powtarzanego na każdej stronie

    Bezpieczeństwo publiczne & Monitorowanie ulic, placów, obiektów strategicznych; automatyczne wykrywanie zagrożeń i incydentów.\\
    \hline
    Transport i logistyka & Monitoring lotnisk, dworców, portów; analiza przepływu pasażerów; automatyczne rozpoznawanie tablic rejestracyjnych.\\
    \hline
    Przemysł & Kontrola procesów produkcyjnych, wykrywanie awarii maszyn, nadzór nad pracownikami i bezpieczeństwem pracy.\\
    \hline
    Handel detaliczny & Zapobieganie kradzieżom, analiza zachowań klientów, optymalizacja układu sklepu.\\
    \hline
    Edukacja & Zwiększanie bezpieczeństwa uczniów i nauczycieli, kontrola dostępu do budynków szkolnych.\\
    \hline
    Ochrona zdrowia & Nadzór nad pacjentami i personelem, zabezpieczenie pomieszczeń szpitalnych, kontrola dostępu do stref wrażliwych.\\
    \hline
    Smart City & Analiza ruchu drogowego, inteligentne sterowanie sygnalizacją świetlną, planowanie urbanistyczne na podstawie danych z kamer.\\
    \hline
\end{longtable}\

Zastosowanie monitoringu wizyjnego opartego na kamerach IP jest obecnie wielosektorowe i dynamiczne. Urządzenia te, integrujące funkcje sensora i procesora danych, stały się podstawą \textbf{systemów analitycznych} w kluczowych obszarach gospodarki i bezpieczeństwa.
W kontekście dalszego rozwoju monitoringu wizyjnego, szczególne znaczenie zyskuje \textbf{sztuczna inteligencja (AI)} i \textbf{uczenie maszynowe (ML)}. Nowoczesne algorytmy pozwalają na automatyczną detekcję zagrożeń, eliminację fałszywych alarmów oraz identyfikację i śledzenie obiektów w czasie rzeczywistym. Integracja tych zaawansowanych technik z \textbf{otwartym oprogramowaniem} — co jest celem niniejszej pracy — otwiera drogę do stworzenia bardziej zaawansowanych, konfigurowalnych i niezależnych narzędzi wspierających bezpieczeństwo oraz analitykę zdarzeń.
Rozwój kamer IP, szczególnie w kontekście inteligentnego monitoringu, jest ściśle powiązany z ewolucją \textbf{Narzędzi Kognitywnych (Cognitive Tools)}.

Narzędzia kognitywne w monitoringu wizyjnym działają na zasadzie mechanizmów inferencji, które imitują procesy decyzyjne i percepcyjne ludzkiego mózgu. Umożliwiają one systemom na przechodzenie od prostej detekcji ruchu do \textbf{zrozumienia kontekstu i intencji} obserwowanych zdarzeń \cite{Fan:Stanford:2020}. Dzięki temu, system monitorujący może automatycznie filtrować szum wizualny i koncentrować uwagę na zdarzeniach o wysokim prawdopodobieństwie zagrożenia lub anomalii. Technologie te transformują surowe dane wideo w zorganizowane i użyteczne metadane, co jest fundamentalne dla automatyki i bezpieczeństwa.
W przeciwieństwie do tradycyjnej detekcji ruchu opartej na różnicy pikseli (algorytm \textit{frame differencing}), nowoczesne systemy wizyjne wykorzystują głębokie sieci neuronowe (DNN) do zaawansowanej analizy obrazu. Pozwala to na realizację funkcji inżynierskich o wysokiej wartości:
\\
Efektywna Analiza Danych wymaga interoperacyjności. Jest to główny powód, dla którego w niniejszej pracy inżynierskiej dąży się do uwolnienia strumienia danych z kamery Tapo C200. Tylko otwarty dostęp do strumienia wideo i metadanych umożliwia ich integrację z zaawansowanymi platformami analitycznymi (np. platformy IoT, systemy Business Intelligence), co jest niemożliwe w zamkniętych ekosystemach producentów.
Kamera IP, połączona z narzędziami kognitywnymi (AI), przestaje być pasywnym urządzeniem rejestrującym, a staje się aktywnym sensorem generującym \textbf{metadane strukturalne}.
 W kontekście systemów Big Data, strumień wideo jest przetworzony w chmurze lub na urządzeniu brzegowym (\textit{edge computing}) na użyteczne informacje, takie jak:
\begin{enumerate}
    \item Liczba wykrytych obiektów (ludzie, pojazdy), ich zagęszczenie (tzw. \textit{heatmaps}) oraz czas przebywania w określonej strefie (np. w handlu detalicznym) znane jako \textbf{dane statystyczne}.
    \item Analiza ścieżek ruchu, wykrywanie nietypowych wzorców zachowania (np. bieganie w strefie zakazu, pozostawienie bagażu) oraz trendów sezonowych w natężeniu ruchu znane jako \textbf{dane behawioralne}.
    \item Na podstawie historycznych i bieżących danych, systemy AI mogą przewidywać prawdopodobne incydenty. Przykładowo, zwiększone zagęszczenie osób w metrze w połączeniu z nietypowymi wzorcami ruchu może wygenerować alarm o potencjalnym zatorze lub wypadku, zanim ten nastąpi, nazywane \textbf{analityką predykcyjną}.
\end{enumerate}


\subsubsection{Monitoring}
\label{subsubsec:monitoring}

Podstawowym i historycznym zastosowaniem kamery IP jest \textbf{nadzór wizyjny (monitoring)}. W odróżnieniu od analogowego CCTV, monitoring oparty na protokole internetowym umożliwia przesyłanie strumienia wideo wysokiej rozdzielczości (np. 1080p w Tapo C200) oraz metadanych poprzez standardowe sieci LAN/WLAN. Z technicznego punktu widzenia, monitoring realizowany jest poprzez ciągłe kodowanie wideo (standardy H.264/H.265), strumieniowanie za pomocą protokołów czasu rzeczywistego (\textbf{RTSP}) oraz zapis cyfrowy na nośnikach lokalnych (microSD, serwer NVR) lub w chmurze. Zaawansowane funkcje, takie jak \textbf{PTZ (Pan-Tilt-Zoom)}, dają inżynierom możliwość dynamicznego dostosowania pola widzenia i śledzenia obiektów bez ingerencji fizycznej, co jest kluczowe w monitorowaniu dużych obszarów (np. hal magazynowych) \cite{Al-Fuqaha:2015}.

\subsubsection{Kontrola Dostępu}
\label{subsubsec:kontrola_dostepu}

Kamery IP są coraz częściej integrowane z systemami \textbf{Kontroli Dostępu (Access Control Systems - ACS)}. Ich rola wykracza poza zwykłe weryfikowanie tożsamości. Dzięki wykorzystaniu AI, kamery stają się kluczowym sensorem w bezdotykowej autoryzacji. Przykłady zastosowań inżynierskich obejmują:
\begin{enumerate}
    \item \textbf{Rozpoznawanie Twarzy (Facial Recognition):} Zastosowanie algorytmów głębokiego uczenia do identyfikacji i weryfikacji osób uprawnionych, automatycznie odblokowując wejścia.
    \item \textbf{Rozpoznawanie Tablic Rejestracyjnych (ANPR):} Automatyczne zezwalanie na wjazd pojazdów do strzeżonych stref (np. parkingów pracowniczych) na podstawie analizy obrazu z kamery.
\end{enumerate}
Takie rozwiązania minimalizują ryzyko błędów ludzkich i zwiększają bezpieczeństwo poprzez ciągłe logowanie zdarzeń wejścia i wyjścia, stanowiąc integralną część zabezpieczeń fizycznych i sieciowych \cite{Bou-Harb:2024}.

\subsubsection{Zarządzanie Procesami Biznesowymi}
\label{subsubsec:zarzadzanie_procesami_biznesowymi}

Wykorzystanie kamer IP w zarządzaniu procesami (\textbf{Business Process Management - BPM}) koncentruje się na optymalizacji operacyjnej poprzez zbieranie danych o efektywności i bezpieczeństwie pracy. W sektorach takich jak produkcja i logistyka, kamery są używane do:
\begin{enumerate}
    \item \textbf{Kontroli Jakości (Quality Assurance - QA):} Monitorowanie linii produkcyjnych w celu automatycznego wykrywania defektów, niezgodności montażu lub nieprawidłowej sekwencji działań.
    \item \textbf{Optymalizacji Przepływu Pracy (\textit{Workflow Optimization}):} Analiza ścieżek ruchu pracowników i pojazdów w celu identyfikacji wąskich gardeł w magazynach i centrach dystrybucyjnych.
\end{enumerate}
Te zastosowania wymagają wysokiej precyzji metadanych i niskiego opóźnienia, co stawia wysokie wymagania przed \textbf{algorytmami analizy brzegowej (Edge Analytics)}, które muszą działać na poziomie procesora kamery lub serwera lokalnego \cite{Abdalla:2020}.

\subsubsection{Technologie Smart}
\label{subsubsec:technologie_smart}

Kamery IP są fundamentalnym elementem \textbf{ekosystemów Smart Home i Smart City}. W tych kontekstach, kamera pełni rolę czujnika behawioralnego, dostarczając danych do zautomatyzowanych systemów decyzyjnych. W budownictwie inteligentnym, Tapo C200, podobnie jak inne urządzenia IoT, jest zintegrowana za pomocą protokołów API z platformami takimi jak \textbf{Google Assistant i Amazon Alexa} (jak wskazano w dokumentacji Tapo). Przykłady zastosowań to:
\begin{enumerate}
    \item \textbf{Automatyzacja Zdarzeniowa:} Detekcja ruchu lub dźwięku (np. wykrywanie płaczu dziecka w Tapo C200) uruchamia inne urządzenia (np. włącza światło, wysyła alert do systemu zarządzania domem).
    \item \textbf{Zarządzanie Energią:} Wykrycie braku obecności osób w pomieszczeniu może prowadzić do automatycznego obniżenia temperatury lub wyłączenia niepotrzebnych urządzeń, przyczyniając się do zwiększenia efektywności energetycznej.
\end{enumerate}
Ten obszar ilustruje potrzebę \textbf{interoperacyjności}, która jest blokowana przez zamknięte protokoły chmurowe, co stanowi główną motywację dla niniejszej pracy inżynierskiej.

\subsubsection{Analiza Danych}
\label{subsubsec:analiza_danych}

Kamera IP, połączona z narzędziami kognitywnymi (AI), przestaje być pasywnym urządzeniem rejestrującym, a staje się aktywnym sensorem generującym \textbf{metadane strukturalne}. W kontekście systemów Big Data, strumień wideo jest intensywnie przetwarzany, stanowiąc bazę dla analityki w czasie rzeczywistym i prognozowania zdarzeń.

Efektywne wykorzystanie danych wizyjnych do celów analitycznych obejmuje trzy główne poziomy inżynierskie:
\begin{enumerate}
    \item \textbf{Ekstrakcja Danych Statystycznych:} Dotyczy pomiarów ilościowych, takich jak gęstość obiektów, liczenie przepływu (\textit{flow counting}) oraz generowanie map ciepła (\textit{heatmaps}) \cite{Minerva:2021}.
    \item \textbf{Analiza Behawioralna i Wzorce Trendów:} Identyfikacja nietypowych sekwencji zdarzeń, które mogą sugerować incydent bezpieczeństwa (np. pozostawiony pakunek) \cite{Al-Fuqaha:2015}.
    \item \textbf{Analityka Predykcyjna (Predictive Analytics):} Przewidywanie potencjalnych przyszłych zdarzeń na podstawie historycznych i bieżących metadanych. Wymaga to integracji i walidacji danych z wielu źródeł IoT \cite{Alaba:2017}.
\end{enumerate}

Możliwość pełnej i niezależnej \textbf{Analizy Danych (Data Analytics)} jest ściśle powiązana z problemem \textit{vendor lock-in}. Uwolnienie strumienia z kamery Tapo C200 jest podstawowym warunkiem inżynierskim dla realizacji zaawansowanej analityki danych.
\subsection{Budowa}
\label{subsec:budowa}
\subsubsection{Budowa Fizyczna - Hardware}
\label{subsubsec:hardware}
\begin{itemize}
    \item Matryca
    \item Mikrofon
    \item Zasilanie
    \item Układ scalony (SoC)
    \begin{itemize}
        \item CPU
        \item Pamięć
        \item Network
    \end{itemize}
\end{itemize}

\subsubsection{Oprogramowanie - Software}
\label{subsubsec:firmware}

\subsection{Zasada działania}
\label{subsec:zasada_dzialania}
\subsubsection{Zarządzanie Zasilaniem}
\label{subsubsec:zarzadzanie_zasilaniem}
\begin{itemize}
    \item POE
    \item Zasilanie
\end{itemize}

\subsubsection{Komunikacja}
\label{subsubsec:komunikacja}
\begin{itemize}
    \item Wifi
    \item HTTP
\end{itemize}

\subsubsection{Provisioning}
\subsubsection{Proces przetwarzania obrazu}
\subsubsection{Proces przetwarzania dźwięku}
\subsubsection{Streamowanie}

\subsection{Funkcje}
\label{subsec:funkcje}
\subsubsection{Obrót PTZ}
\subsubsection{Wykrywanie obiektów i zdarzeń - AI}
\subsubsection{Wykrywanie ruchu}
\subsubsection{Noktowizja i termowizja}
\subsubsection{Dwukierunkowe audio}
\subsubsection{Zapis danych}
\subsubsection{Integracja z Inteligentnymi Systemami}
\subsubsection{Powiadomienia push}

\subsection{Ograniczenia}
\label{subsec:ograniczenia}
Ograniczenia wynikające z technologii i modelu biznesowego producentów.
\subsubsection{Przepustowość i zużycie danych}
\subsubsection{Zależność od sieci}
\subsubsection{Bezpieczeństwo}
\subsubsection{Zależność od producenta / Chmura}

\subsection{Wnioski - Analiza}
\label{subsec:wnioski_analiza}

\newpage
\section{Rozdział 2}\label{chap:drugi}
Rozdział 2 

\newpage
\section{Opis przeprowadzonych testów, ich zakres i wyniki}
\label{chap:testy}
Przeprowadzenie kompleksowych testów jest nieodłącznym elementem każdej pracy naukowej i inżynierskiej. Dzięki testom możemy weryfikować skuteczność, efektywność oraz bezpieczeństwo opracowanego rozwiązania. Metody i rodzaje ewentualnych mierzalnych testów opisano w rozdziale ,,Cel i zakres pracy''. Rozdział ten zależeć będzie od podjętego tematu i zaleceń promotora. Poniższe podrozdziały są tylko przykładem.

\subsection{Zakres testów}

W ramach badań przeprowadzono trzy główne kategorie testów\ldots

\subsection{Wyniki testów}

Na początek, w krótkim wstępie, powinniśmy przedstawić charakterystykę testów oraz przypomnieć czytelnikowi cele i założenia, które zostały wcześniej opisane w rozdziale poświęconym metodologii testowania. Następnie warto przedstawić specyfikację środowiska testowego, uwzględniając zarówno specyfikację sprzętową, jak i programową, a także szczegółowe wersje używanego oprogramowania czy systemów operacyjnych.

Kolejnym krokiem jest podsumowanie wszystkich przeprowadzonych testów, wskazując, ile z nich zakończyło się sukcesem, a ile niepowodzeniem. Szczegółowe wyniki powinny zawierać informacje na temat funkcjonalności, wydajności oraz bezpieczeństwa. Jeśli przeprowadzono testy użyteczności, warto również dodać informacje na temat feedbacku od użytkowników oraz napotkanych trudności.

Niezbędne są też wizualizacje: wykresy, diagramy czy tabele, które pomogą czytelnikowi zrozumieć i zinterpretować uzyskane dane. Po prezentacji wyników ważna jest ich głęboka analiza, interpretacja w kontekście wcześniej postawionych celów oraz omówienie ewentualnych odchyleń od oczekiwań. Pamiętaj jednak, że sam wykres to za mało. Należy wskazać czytelnikowi, które aspekty uzyskanych wyników potwierdzają tezy pracy.

\section{Analiza aspektów prawnych}

W kontekście pracy inżynierskiej, analiza aspektów prawnych odgrywa kluczową rolę. Oprogramowanie, protokoły, algorytmy i różne metody, które zostały opracowane, podlegają różnym normom prawnym, które muszą być rozważone, aby zapewnić legalność oraz etyczność opracowanego rozwiązania.

\subsection{Prawo autorskie}

Wszystkie skrypty, kody źródłowe oraz dokumentacje stworzone w ramach tej pracy są chronione prawem autorskim. Jest to kluczowe, aby zabezpieczyć prawa własności intelektualnej autora oraz zapewnić odpowiednie licencjonowanie. Należy zastanowić się nad odpowiednią licencją, która umożliwi innym korzystanie z opracowanego rozwiązania, jednocześnie chroniąc prawa autora.

\subsection{Ochrona danych}

Jeśli w pracy zostały wykorzystane jakiekolwiek dane, szczególnie te dotyczące osób fizycznych, istotne jest przestrzeganie przepisów związanych z ochroną danych osobowych. W kontekście europejskim, należy zwrócić uwagę na zgodność z Rozporządzeniem o Ochronie Danych Osobowych (RODO).

\subsection{Prawo patentowe}

Jeśli opracowane rozwiązania są nowatorskie i mają potencjał komercyjny, warto rozważyć możliwość ubiegania się o patent. Chociaż proces patentowania w obszarze informatyki może być skomplikowany, chroni on przed nieautoryzowanym wykorzystaniem wynalazku.

\subsection{Licencjonowanie oprogramowania}

W zależności od charakteru rozwiązania oraz planowanej dystrybucji, istotne jest wybór odpowiedniej licencji. Może to być licencja open-source, która pozwala na szerokie wykorzystanie kodu przez społeczność, lub licencja komercyjna, która ogranicza dostęp do kodu i jego wykorzystanie.

\subsection{Etyka w nauce}

Oprócz ścisłych przepisów prawnych, ważne jest również przestrzeganie ogólnych zasad etyki w nauce. Obejmuje to uczciwość, transparentność oraz odpowiedzialność za stworzone rozwiązania i ich ewentualne wpływy na społeczeństwo.

Wnioski z niniejszego rozdziału mają na celu podkreślenie znaczenia aspektów prawnych w~procesie tworzenia nowych rozwiązań w dziedzinie informatyki stosowanej. Prawo w tej dziedzinie jest skomplikowane i dynamicznie się zmienia, dlatego istotne jest ciągłe monitorowanie aktualnych przepisów i dostosowywanie do nich opracowywanych rozwiązań.



\newpage
\section*{Wnioski Końcowe}
\addcontentsline{toc}{section}{Wnioski Końcowe}

Niniejsza praca inżynierska zrealizowała postawiony cel główny, tworząc kompletne i modułowe rozwiązanie Open Source dla kamer TP-Link Tapo C200. Projekt udowodnił, że bariera \textbf{vendor lock-in} może być skutecznie przełamana za pomocą inżynierii wstecznej protokołów własnościowych i integracji sprawdzonych narzędzi (Docker, FFmpeg, OpenCV).

Osiągnięte wyniki potwierdzają:
\begin{enumerate}
    \item Pełną i stabilną kontrolę PTZ z poziomu autorskiego interfejsu.
    \item Możliwość lokalnego streamingu wideo z minimalnymi opóźnieniami.
    \item Skuteczną implementację autorskiej detekcji ruchu, oferującej większą konfigurowalność niż wbudowane funkcje kamery.
\end{enumerate}

\subsection*{Kierunki dalszego rozwoju}
\addcontentsline{toc}{subsection}{Kierunki dalszego rozwoju}

Dalsze prace mogą koncentrować się na:
\begin{itemize}
    \item Integracji z protokołami Smart Home (np. MQTT) w celu komunikacji z systemami takimi jak Home Assistant.
    \item Zastosowaniu WebRTC (Web Real-Time Communication) w celu dalszej redukcji opóźnień streamingu.
    \item Wdrożeniu lekkich modeli uczenia maszynowego (np. YOLOv5 Nano) dla zaawansowanej detekcji obiektów, wykorzystujących akcelerację sprzętową GPU hosta.
\end{itemize}

\subsection*{Podsumowanie pracy}
\addcontentsline{toc}{subsection}{Podsumowanie pracy}

Opracowany prototyp stanowi w pełni funkcjonalną i otwartą alternatywę dla zamkniętego ekosystemu producenta, oferując użytkownikowi pełną suwerenność nad gromadzonymi danymi i możliwością rozbudowy systemu.

\newpage

\section*{Wyjaśnienie skrótów używanych w pracy}
Jeśli w pracy używana jest znaczna ilość skrótów, symboli, lub nieznanych powszechnie terminów, to warto w tym miejscu zrobić ich słownik.

% Wykaz literatury
\newpage

\printbibliography

% Inne spisy

\newpage
\pagestyle{fancy}
\listoffigures
\clearpage
\newpage
\listoftables
\newpage
\listofmyequations
\newpage
\renewcommand{\lstlistlistingname}{Spis kodów}
\lstlistoflistings

\newpage
\input{wykazy.tex}

\end{document}
