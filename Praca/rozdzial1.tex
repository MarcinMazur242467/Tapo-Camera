\section{Tytuł rozdziału pierwszego}
\label{chap:pierwszy}
Rozdział ten stanowi techniczne wprowadzenie do zagadnienia, oparte na analizie literatury i istniejących rozwiązań inżynierskich. Autor prezentuje w nim najważniejsze badania, aplikacje oraz koncepcje techniczne związane z tematem pracy inżynierskiej. Analiza literatury ma na celu przybliżenie czytelnikowi kluczowych aspektów technicznych, identyfikację potencjalnych luk w dostępnych rozwiązaniach oraz podkreślenie znaczenia innowacji proponowanych w~ramach tej pracy. Wprowadzenie do rozdziału podkreśla, dlaczego dokładne zrozumienie literatury i istniejących technologii jest kluczowe dla sukcesu projektu inżynierskiego. Autor wyjaśnia, jak analiza literatury wpływa na kształtowanie podejścia do projektu oraz jakie korzyści niesie dla kompleksowego zrozumienia badanego problemu w kontekście inżynierskim.

\subsection{Tytuł podrozdziału rozdziału}
W tym fragmencie koncentrujemy się na przedstawieniu kluczowych koncepcji oraz terminologii, które są istotne w kontekście analizy technicznej omawianego zagadnienia. Wskazane jest, aby autor dokładnie zdefiniował pojęcia techniczne i inżynierskie, które będą używane w~dalszej części pracy, zarówno podczas analizy literatury, jak i podczas prezentacji oraz interpretacji wyników własnych badań lub projektu inżynierskiego. Na przykład, jeżeli praca skupia się na analizie i projektowaniu rozwiązań opartych o sztuczną inteligencję w przemyśle, konieczne jest dokładne zdefiniowanie takich terminów jak ,,algorytmy uczenia maszynowego'', ,,sieci neuronowe'' czy ,,automatyzacja procesów przemysłowych''.

\subsubsection{Tytuł podpodrozdziału rozdziału}
Nie planujmy zbytnio zagłębiać się poniżej tego poziomu nagłówków.

\subsection{Tytuł podrozdziału rozdziału}
W niniejszym podrozdziale przedstawiona zostanie analiza oraz uzasadnienie wyboru konkretnych technologii i narzędzi, które zostaną użyte w ramach realizacji tematu pracy. Wybór technologii ma kluczowe znaczenie dla efektywności, jakości oraz wiarygodności wyników, dlatego niezbędne jest dogłębne uzasadnienie dokonanych wyborów.
Przechodząc do analizy i uzasadnienia wyboru konkretnych technologii, warto przyjrzeć się dostępnym opcjom na rynku. W tym celu zostaną przedstawione główne technologie dostępne dla realizacji tematu pracy, zarówno te powszechnie stosowane, jak i te mniej znane, ale posiadające pewne unikatowe cechy.
Przy wyborze technologii i narzędzi kluczowe jest ustalenie kryteriów, które pomogą w~podjęciu decyzji. 
\newpage
Do głównych kryteriów można zaliczyć m.in.:
\begin{itemize}
    \item dostępność i wsparcie techniczne,
    \item wydajność i skalowalność,
    \item bezpieczeństwo i niezawodność,
    \item koszty zakupu i utrzymania,
    \item zgodność z wymaganiami pracy.
\end{itemize}

W tym miejscu warto również odnieść się do wymagań zdefiniowanych w celu pracy oraz uwzględnić analizę źródeł informacji, na której bazują nasze wybory.

Po dokładnej analizie dostępnych technologii zostanie przedstawione uzasadnienie wyboru konkretnych rozwiązań. Wyjaśnione zostaną powody, dla których wybrane narzędzia i technologie są najbardziej odpowiednie do realizacji tematu pracy. Warto również uwzględnić opinie ekspertów oraz odwołać się do recenzowanych źródeł informacji. Nawet jeśli pewne technologie były znane wcześniej, warto jest rozważyć alternatywne rozwiązania. Przedstawienie alternatyw pozwoli na lepsze zrozumienie zalet i wad wybranej technologii. Umożliwi to również lepsze przygotowanie do potencjalnych problemów oraz zidentyfikowanie obszarów, w których wybrana technologia może wymagać dodatkowego wsparcia czy rozszerzeń.

\subsubsection{Tytuł podpodrozdziału rozdziału}
Nie planujemy zbytnio zagłębiać się poniżej tego poziomu nagłówków.
\subsection{Tytuł podrozdziału rozdziału}
W końcowej części podrozdziału podsumowane zostaną główne wnioski płynące z analizy oraz wyboru technologii i narzędzi. Uzasadnienie będzie służyć jako fundament dla dalszej części pracy, a także wsparcie dla czytelnika w zrozumieniu kluczowych decyzji podjętych przez autora w kontekście realizacji tematu pracy. Przy tworzeniu tego podrozdziału warto pamiętać o krytycznym podejściu do źródeł informacji oraz o zapewnieniu, że przedstawione argumenty są oparte na wiarygodnych podstawach, tak aby całość była spójna i wiarygodna. Należy podchodzić tu z pewną dozą nieufności do źródeł i zweryfikować ich wiarygodność.
Szczególną nieufność zaleca się zachować w stosunku do odpowiedzi systemów AI (ChatGPT, Bard, itp.). Są one aż nadto kreatywne i potrafią przekonująco opisywać rzeczy nieistniejące.

\subsection{Odwołania do literatury i sposób cytowania}
W pracy inżynierskiej należałoby odwoływać się do różnych rodzajów literatury, które wspierają Twoje działania, analizę i argumentację. Pamiętaj, aby korzystać ze źródeł, zwracając uwagę na ich aktualność, renomę autorów oraz miejsce publikacji. Wskazane jest, aby opierać się głównie na źródłach naukowych, które są wiarygodne i posiadają recenzję naukową. 
Zachowaj jednolity styl typu autor-data w całej pracy, np. styl harwardzki lub APA (zalecane korzystanie z~narzędzia do zarządzania bibliografią Mendeley https://www.mendeley.com), 
Polecane źródła to:
\begin{enumerate}
    \item Źródła naukowe i badawcze: artykuły naukowe, książki, monografie i raporty z badań publikowane w renomowanych czasopismach naukowych lub wydawnictwach akademickich stanowią główne źródła w pracy inżynierskiej, tworząc podstawę teoretyczną i empiryczną.
    \item Podręczniki i książki akademickie: książki napisane przez ekspertów w danej dziedzinie dostarczają pogłębionej wiedzy i uzupełniają teoretyczne tło pracy.
    \item Artykuły konferencyjne: artykuły prezentowane na konferencjach naukowych często zawierają nowe badania i innowacyjne podejścia, które mogą odnosić się do Twojego tematu.
    \item Raporty instytucji i organizacji: jeśli praca ma aspekt praktyczny, warto korzystać z~raportów i publikacji wydawanych przez instytucje lub organizacje związane z badanym obszarem.
    \item Źródła statystyczne: w przypadku badań opartych na danych statystycznych, odwołaj się do źródeł takich jak rządowe raporty statystyczne, bazy danych lub publikacje statystyczne.
    \item Materiały internetowe: odpowiednio zweryfikowane i renomowane źródła internetowe mogą dostarczyć aktualnych informacji czy opinii ekspertów (pamiętaj o wiarygodności takich źródeł).
    \item Literatura praktyczna: jeśli praca dotyczy konkretnych technologii, narzędzi czy działań praktycznych, korzystaj z dokumentacji, przewodników użytkownika oraz innych materiałów związanych z praktyką.
    \item Literatura interdyscyplinarna: w niektórych przypadkach może być konieczne odwoływanie się do literatury z innych dziedzin, jeśli ma to wpływ na Twój temat.
\end{enumerate}
\newpage
Poniżej przedstawiono wykaz literatury, który znajduje powinien znaleźć się na końcu pracy, w układzie alfabetycznym według nazwisk autorów:

\begin{itemize}
    \item \textbf{Książka:}
    \begin{itemize}
        \item Autor A. B., Data wydania, \emph{Tytuł książki}, Miejsce wydania, Wydawnictwo, ISBN lub DOI, jeśli dostępny.
    \end{itemize}
    
    \item \textbf{Rozdział w książce:}
    \begin{itemize}
        \item Autor C. D., Data wydania, \emph{Tytuł rozdziału}, w: Autor E. F. (red.), \emph{Tytuł książki}, Zakres stron rozdziału, Miejsce wydania, Wydawnictwo, ISBN lub DOI, jeśli dostępny.
    \end{itemize}
    
    \item \textbf{Artykuł:}
    \begin{itemize}
        \item Autor G. H., Data wydania, \emph{Tytuł artykułu}, \emph{Tytuł czasopisma}, Numer, Zakres stron artykułu, DOI, jeśli dostępny.
    \end{itemize}
    
    \item \textbf{Dokumenty elektroniczne:}
    \begin{itemize}
        \item Autor I. J., Data wydania, \emph{Tytuł dokumentu}, Ścieżka dostępu, Data dostępu, ISBN lub DOI, jeśli dostępny.
    \end{itemize}
    
    \item \textbf{Strony WWW:}
    \begin{itemize}
        \item Tytuł strony, Data publikacji, Tytuł serwisu, Ścieżka dostępu, Data dostępu.
    \end{itemize}
\end{itemize}

Ten kod generuje sekcję ,,Wykaz literatury'' na końcu pracy, w której znajduje się wykaz literatury zgodnie z Twoimi wskazówkami, sformatowany jako lista punktowana. Warto zauważyć, że użyto komendy \emph{} do wyróżnienia tytułów, aby zachować ich kursywę. Możesz dostosować ten kod do swoich potrzeb, wprowadzając właściwe dane do każdego elementu.

Poniżej przykład w jaki sposób można odwołać się do literatury:

Odkrywanie nowych perspektyw w dziedzinie sztucznej inteligencji (SI) i gier komputerowych jest związane z dynamicznym rozwojem technologii i teorii. \cite{Smith:2019} wskazuje, że wykorzystanie zaawansowanych algorytmów uczenia maszynowego może znacząco poprawić jakość interakcji człowieka z grami. Istnieją również dowody na to, że wprowadzenie elementów SI, takich jak systemy generowania proceduralnego, może znacznie wzbogacić doświadczenie gracza \citep{Jones:2020}. Jednakże, ważne jest, aby pamiętać o etycznych implikacjach związanych z wykorzystaniem SI w grach \citep{Sizer:2018,Jones:2020}. Wnioski z badań nad tą tematyką mogą przyczynić się do dalszego rozwoju zarówno dziedziny gier komputerowych, jak i nauki o sztucznej inteligencji.

W powyższym kodzie \textbackslash{}cite\{\} służy do wstawienia odnośnika do pozycji z wykazu literatury bez nawiasów, jeśli odniesienie jest fragmentem zdania. Jeśli jest to zwykły odnośnik (taki, który może być usunięty ze zdania bez zmiany jego składni), to powinien być wpisany w~nawiasach okrągłych przy pomocy znacznika \textbackslash{}citep\{\}. Odnośniki zostaną automatycznie sformatowane jako nazwiska autorów oraz lata wydania. Pamiętaj, że musisz mieć odpowiednie wpisy w pliku .bib zdefiniowane w Twoim projekcie LaTeX, aby to działało poprawnie.

\subsection{Rysunki}
Pamiętaj, że jeden obraz jest więcej wart niż tysiące słów. Rysunki powinny mieć jasny cel i~służyć konkretnym funkcjom. Mogą one ułatwiać zrozumienie trudnych koncepcji, prezentować dane numeryczne, ilustrować procesy, czy przedstawiać schematy czy grafiki. Rysunki powinny być czytelne i przejrzyste. Unikaj nadmiernego natłoku informacji, a także zbyt małych fontów lub detali, które mogą utrudnić odczyt.

Każdy rysunek powinien mieć unikalny numer oraz odpowiedni opis. Numeracja i opisy umożliwiają czytelnikowi łatwiejsze odnajdywanie konkretnych rysunków i ich zrozumienie. Zgodnie z wymogami PŁ podpis pod rysunkiem ma być:
\begin{itemize}
    \item justowany do lewej strony lub wyśrodkowany, font rozmiar 10 pkt,
    \item podpis ,,Rys.'' oznaczyć numerem podającym rozdział i numer kolejny rysunku w rozdziale lub zachowując ciągłość numeracji w całej pracy,
    \item źródło obok podpisu rysunku, z zachowaniem jednolitego stylu odwołań do źródeł w~całej pracy.
\end{itemize}

Rysunki powinny być ściśle związane z tekstem, który je otacza. Wyjaśnij, jakie aspekty rysunku odnoszą się do omawianej treści. Jeśli korzystasz z rysunków stworzonych przez innych autorów, zawsze podaj źródło i uzyskaj odpowiednie pozwolenia na ich użycie. Rysunki powinny być oryginalne lub użyte zgodnie z zasadami praw autorskich.

Styl i język rysunków powinny być spójne w całej pracy. Wybierz jeden styl graficzny (np. kształty, kolory, linie) i trzymaj się go, aby zachować spójność wizualną. Nie przesadzaj z ilością rysunków. Użyj ich tam, gdzie naprawdę przyczyniają się do zrozumienia treści lub udowodnienia argumentu. Jeśli prezentujesz dane numeryczne, wykorzystaj odpowiednie wykresy, które czytelnie ukazują relacje między danymi.

Upewnij się, że rysunki są w odpowiedniej rozdzielczości, aby były wyraźne w wydruku. Wybierz format pliku, który zachowa jakość obrazu, na przykład PNG, JPEG lub PDF. Nie przesadzaj też z rozdzielczością obrazów, tak by praca nie ,,ważyła zbyt dużo''.

Odniesienia do rysunków mogą pełnić funkcję części zdania, wtedy nie używa się nawiasów (przykład rysunku \ref{fig:nazwa_rysunku} w~następnym zdaniu), albo mogą być wtrącone do zdania, wtedy używa się nawiasów (przykład: ,,(rys.~\ref{fig:rys2})'') w~następnym zdaniu. Rysunki powinny być podpisane ze wskazaniem na źródło:
 W nawiasach ( ) -- jeśli rysunek~\ref{fig:nazwa_rysunku} został zacytowany ze źródła w~bibliografii.
W przypisie dolnym bezpośredni link do rysunku -- jeśli rysunek został zacytowany ze strony internetowej, której nie ma w bibliografii (rys.~\ref{fig:rys2}),
W nawiasach ,,(opracowanie własne)'' (rys.~\ref{fig:rys3}).

W LaTeX możesz wstawiać rysunki za pomocą środowiska figure oraz komendy ,,includegraphics{}''. Aby dodać podpisy oraz źródła do rysunków, możesz użyć komendy ,,caption{}''. Możesz także użyć przypisów dolnych do rysunków za pomocą komendy ,,footnote{}''. Poniżej przedstawiona została instrukcja, jak to zrobić:

Wstaw rysunek wewnątrz środowiska figure i użyj komendy includegraphics{} do umieszczenia obrazu.

Aby wymusić, aby dany rysunek był umieszczony w danym rozdziale w LaTeXu, możesz użyć opcji [H] dostarczonej przez pakiet float. Ta opcja unika przenoszenia rysunku na inne strony i umieszcza go dokładnie w miejscu, w którym go umieściłeś. 

\begin{figure}[H]
    \centering
    \includegraphics[width=0.9\textwidth]{Figures/1.jpg}
    \caption{Studenci Wydziału FTIMS \citep{strona:FTIMS}}
    \label{fig:nazwa_rysunku}
\end{figure}

\begin{figure}[H]
    \centering
    \includegraphics[width=0.9\textwidth]{Figures/2.png}
    \caption[Dane liczbowe o Politechnice Łódzkiej oraz najważniejsze informacje o uczelni]{Dane liczbowe o Politechnice Łódzkiej oraz najważniejsze informacje o uczelni\protect\footnotemark}
    \label{fig:rys2}
\end{figure}
\footnotetext{Źródło: https://p.lodz.pl/uczelnia/pl-w-liczbach}

\begin{figure}[H]
    \centering
    \includegraphics[width=0.9\textwidth]{Figures/3.png}
    \caption{Postać ,,Ninjago'' (opracowanie własne).}
    \label{fig:rys3}
\end{figure}

Upewnij się, że masz odpowiednie obrazy w lokalizacji i formatach (np. PNG, JPEG) wskazanych w komendzie ,,includegraphics{}''. Sprawdź, czy są one w odpowiedniej rozdzielczości.
Zaktualizuj opisy, źródła i ścieżki dostępu do rysunków zgodnie z Twoimi potrzebami.
Skompiluj dokument LaTeX, aby zobaczyć efekty.
Dzięki tym krokom, będziesz w stanie wstawiać i~podpisywać rysunki w swoim dokumencie LaTeX w sposób czytelny i zgodny ze standardami.

\subsection{Wzory}
Wzory matematyczne odgrywają istotną rolę w pracy inżynierskiej, szczególnie jeśli temat związany jest z naukami ścisłymi lub technicznymi. Odpowiednie formatowanie wzorów ma kluczowe znaczenie, aby zachować czytelność i klarowność pracy. Poniżej przedstawione są wytyczne dotyczące wzorów w pracy inżynierskiej:

\begin{enumerate}
  \item \textbf{Wyśrodkowanie:} Wzory matematyczne powinny być wyśrodkowane na stronie, co zapewnia równomierne rozmieszczenie względem tekstu i estetyczny wygląd.

  \item \textbf{Numeracja przy prawym marginesie:} Numeracja wzorów powinna znajdować się przy prawym marginesie strony. To ułatwia odnalezienie konkretnego wzoru dla czytelnika.

  \item \textbf{Numeracja w nawiasie okrągłym:} Numeracja wzorów powinna być umieszczona w~nawiasie okrągłym bezpośrednio przy wzorze. Na przykład: (1), (2), (3), ... Ta numeracja jest ciągła w całej pracy.

  \item \textbf{Spójna numeracja:} Wzory powinny być numerowane ciągle w całej pracy inżynierskiej, niezależnie od rozdziałów czy sekcji. Każdy nowy wzór otrzymuje kolejny numer.
\end{enumerate}

Przykład poprawnie sformatowanego wzoru w pracy inżynierskiej:
\begin{equation}
    ax^2 + bx + c = 0,\label{eq:kwadratowe}
\end{equation}
\noindent %MM: ponieważ to jest jedno zdanie
gdzie $a$, $b$ i $c$ są współczynnikami tego równania kwadratowego.

Prawidłowe formatowanie wzorów (\ref{eq:kwadratowe}) nie tylko poprawia czytelność, ale również nadaje pracy profesjonalny wygląd. Upewnij się, że stosujesz te wytyczne w każdym miejscu, gdzie pojawiają się wzory w Twojej pracy inżynierskiej.

Oto przykład poprawnie sformatowanego wzoru z macierzą w pracy inżynierskiej:
\begin{equation}
    {\bf A} = \begin{bmatrix} %MM: zmieniłem!
        a & b \\
        c & d
    \end{bmatrix}
    \label{eq:macierz}
\end{equation}

Odwzorowanie macierzowe można opisać za pomocą wzoru (\ref{eq:macierz}).
Przyjęło się, że dużymi wytłuszczonymi literami oznacza się macierze, małymi, ale też wytłuszczonymi -- wektory, kursywą -- skalary. Dotyczy to także liter greckich. Jeśli przyjmiesz inną konwencję - trzeba ją opisać.
    
\subsection{Tabele}
Wytyczne dotyczące formatowania tabel w pracy inżynierskiej:
\begin{enumerate}
    \item Tytuł tabeli: tytuł tabeli powinien znajdować się nad tabelą. Może być wyjustowany do lewej strony lub wyśrodkowany. Rozmiar fontu tytułu to 10 punktów.
    \item Numeracja tabel: tabele powinny być numerowane, z numerem oznaczającym zarówno rozdział, jak i kolejny numer tabeli w danym rozdziale. Dopuszczalne jest prowadzenie numeracji ciągłej tak jak w przypadku rysunków. Numeracja powinna być umieszczona nad tabelą.
    \item Źródło tabeli: źródło tabeli powinno być umieszczone obok podpisu tabeli. Upewnij się, że styl odwołań do źródeł jest spójny z całą pracą. Stosuj ten sam format odwołań do źródeł, który jest wykorzystywany w całym tekście.
\end{enumerate}
\begin{table}[htbp]
    \centering
    \caption{Tabela z przykładowymi danymi (opracowanie własne)}
    \begin{tabular}{|c|c|}
        \hline
        Kolumna 1 & Kolumna 2 \\
        \hline
		\hline
        Wartość 1 & Wartość 2 \\
        \hline
        Wartość 3 & Wartość 4 \\
        \hline
    \end{tabular}
    \label{tab:przyklad-tabeli}
\end{table}

Tabela~\ref{tab:przyklad-tabeli} przedstawia przykładowe dane. Do każdej tabeli analogicznie, jak do rysunków powinno być odniesienie w~tekście. Zgodnie z wytycznymi, pamiętaj o spójności formatowania tabel oraz o tym, że źródła powinny być precyzyjnie wskazane, aby czytelnik mógł zweryfikować źródło danych lub informacji przedstawionych w tabeli.

Proszę zwrócić uwagę, aby etykiety rysunków i tabel zawierały tylko ich opis i legendę, a~nie zawierały żadnych dodatkowych informacji, których miejsce jest w głównym tekście. Jeżeli nagłowek tabeli jest dłuższy niż dwie linie, to jest to znak, że należy się zastanowić, czy nie zawiera ww. informacji.

\subsection{Wskazówki dotyczące formatowania tekstu}
W tekście pracy inżynierskiej ważne jest dbanie o poprawne formatowanie, które wpływa na czytelność i estetykę dokumentu. Oto kilka wskazówek dotyczących formatowania tekstu:

\begin{enumerate}
    \item Unikaj umieszczania wielokrotnych spacji w tekście. Jeśli zauważysz obszar z wieloma spacjami, zamieniaj je na pojedyncze spacje. Możesz to robić wielokrotnie, aż cały tekst będzie wolny od nadmiernych spacji,
    \item upewnij się, że nie ma spacji przed kropkami, przecinkami, średnikami i dwukropkami. Poprawnie to sformatuj, zamieniając ,, ,'' na ,,,'', ,, .'' na ,,.'', ,, ;'' na ,,;'' i ,, :'' na ,,:''.
    \item usuń spacje z prawej strony nawiasu otwierającego i z lewej strony nawiasu zamykającego. Zamień ,,( '' na ,,('' i ,, )'' na ,,)''.
    \item przyimki takie jak ,,w'', ,,o'', ,,i'', ,,z'', ,,u'' i ,,a'' powinny być przyklejone spacją niełamiącą do następnego wyrazu (przykład \LaTeX: ,,w$\sim$tekście''). To oznacza, że jeśli przyimki znajdują się na końcu linii, to cały przyimek przeniesie się do następnej linii razem z~następnym wyrazem,
    \item symbole miar umieszcza się po spacji (w \LaTeX \ należy używać symbolu $\sim$) np. 2~s (dwie sekundy), 12~Mbps (dwanaście magabitów na sekundę), 2023~r. (rok 2023). Nie dotyczy to nielicznych wyjątków, w których tradycyjnie symbol \%, ${}^\circ$ umieszcza się zaraz po liczbie (np. $100\%, 90^\circ$).
\end{enumerate}

\textbf{Przykład:}
\begin{itemize}
    \item Zły: ,,A artykule przedstawiono wyniki , w których nie uwzględniono szybkości renderingu''.
    \item Poprawny: ,,A artykule przedstawiono wyniki, w których nie uwzględniono szybkości renderingu''.
\end{itemize}

Pamiętaj, że poprawne formatowanie tekstu wpływa na profesjonalny wygląd Twojej pracy oraz ułatwia czytelnikowi zrozumienie treści. Przy zachowaniu tych wytycznych tekst staje się bardziej klarowny i estetyczny.

\subsection{Język pracy}
W pracy inżynierskiej jako formie pisemnej o charakterze naukowym ważne jest utrzymanie wysokiego poziomu językowego. Poniżej przedstawionych jest kilka najczęstszych błędów językowych, które warto unikać:

\begin{enumerate}
    \item Błędy w składni, formach czasowników, strukturze zdań itp. mogą wprowadzić zamieszanie i utrudnić zrozumienie treści.
    \item Błędy pisowni stanowią poważne zaburzenie estetyki tekstu i mogą wskazywać na niedbałość w przygotowaniu pracy.
    \item Praca inżynierska powinna być napisana językiem formalnym. Unikaj zbyt luźnych zwrotów, skrótów, kolokwializmów i nieodpowiedniego słownictwa.
    \item Przełączanie się między pierwszą, drugą i trzecią osobą w tekście może wprowadzać zamieszanie. Warto wybrać jedną konkretną osobę gramatyczną i się jej trzymać (preferowane: formy nieosobowe, trzecia osoba lub strona bierna).
    \item Zbyt długie zdania utrudniają zrozumienie treści. Staraj się tworzyć zrównoważone zdania, unikając zdań zbyt skomplikowanych.
    \item Niewłaściwe cytowanie lub nieumieszczenie odnośnika do źródła może prowadzić do zarzutów o plagiat.
    \item Używaj spójnych terminów, symboli i oznaczeń przez całą pracę. Unikaj wieloznaczności i sprzeczności terminologicznych.
    \item Błędy w zapisie wzorów matematycznych mogą prowadzić do nieporozumień. Upewnij się, że wzory są poprawne i czytelne.
    \item Niepoprawne użycie przecinków, kropek, dwukropków i innych znaków przestankowych może wpłynąć na zrozumienie tekstu.
    \item Brak spójności w stylu pisania, nadmierne powtórzenia pewnych wyrażeń czy zbyt długie, zawiłe zdania mogą utrudnić czytanie.
    \item Praca powinna być starannie przeanalizowana pod kątem błędów przed ostatecznym złożeniem. Przeoczone błędy mogą wpłynąć na ogólny odbiór pracy.
\end{enumerate}

Staranność w unikaniu błędów w pracy ułatwi jej zrozumienie i odbiór przez czytelników (pamiętaj, że wśród nich bedzie twój przyszły recenzent). Warto poświęcić czas na redakcję i~korektę, aby uzyskać jak najlepszy efekt.
