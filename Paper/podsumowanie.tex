\section*{Wnioski Końcowe}
\addcontentsline{toc}{section}{Wnioski Końcowe}

Niniejsza praca inżynierska zrealizowała postawiony cel główny, tworząc kompletne i modułowe rozwiązanie Open Source dla kamer TP-Link Tapo C200. Projekt udowodnił, że bariera \textbf{vendor lock-in} może być skutecznie przełamana za pomocą inżynierii wstecznej protokołów własnościowych i integracji sprawdzonych narzędzi (Docker, FFmpeg, OpenCV).

Osiągnięte wyniki potwierdzają:
\begin{enumerate}
    \item Pełną i stabilną kontrolę PTZ z poziomu autorskiego interfejsu.
    \item Możliwość lokalnego streamingu wideo z minimalnymi opóźnieniami.
    \item Skuteczną implementację autorskiej detekcji ruchu, oferującej większą konfigurowalność niż wbudowane funkcje kamery.
\end{enumerate}

\subsection*{Kierunki dalszego rozwoju}
\addcontentsline{toc}{subsection}{Kierunki dalszego rozwoju}

Dalsze prace mogą koncentrować się na:
\begin{itemize}
    \item Integracji z protokołami Smart Home (np. MQTT) w celu komunikacji z systemami takimi jak Home Assistant.
    \item Zastosowaniu WebRTC (Web Real-Time Communication) w celu dalszej redukcji opóźnień streamingu.
    \item Wdrożeniu lekkich modeli uczenia maszynowego (np. YOLOv5 Nano) dla zaawansowanej detekcji obiektów, wykorzystujących akcelerację sprzętową GPU hosta.
\end{itemize}

\subsection*{Podsumowanie pracy}
\addcontentsline{toc}{subsection}{Podsumowanie pracy}

Opracowany prototyp stanowi w pełni funkcjonalną i otwartą alternatywę dla zamkniętego ekosystemu producenta, oferując użytkownikowi pełną suwerenność nad gromadzonymi danymi i możliwością rozbudowy systemu.