\section{Opis przeprowadzonych testów, ich zakres i wyniki}
\label{chap:testy}
Przeprowadzenie kompleksowych testów jest nieodłącznym elementem każdej pracy naukowej i inżynierskiej. Dzięki testom możemy weryfikować skuteczność, efektywność oraz bezpieczeństwo opracowanego rozwiązania. Metody i rodzaje ewentualnych mierzalnych testów opisano w rozdziale ,,Cel i zakres pracy''. Rozdział ten zależeć będzie od podjętego tematu i zaleceń promotora. Poniższe podrozdziały są tylko przykładem.

\subsection{Zakres testów}

W ramach badań przeprowadzono trzy główne kategorie testów\ldots

\subsection{Wyniki testów}

Na początek, w krótkim wstępie, powinniśmy przedstawić charakterystykę testów oraz przypomnieć czytelnikowi cele i założenia, które zostały wcześniej opisane w rozdziale poświęconym metodologii testowania. Następnie warto przedstawić specyfikację środowiska testowego, uwzględniając zarówno specyfikację sprzętową, jak i programową, a także szczegółowe wersje używanego oprogramowania czy systemów operacyjnych.

Kolejnym krokiem jest podsumowanie wszystkich przeprowadzonych testów, wskazując, ile z nich zakończyło się sukcesem, a ile niepowodzeniem. Szczegółowe wyniki powinny zawierać informacje na temat funkcjonalności, wydajności oraz bezpieczeństwa. Jeśli przeprowadzono testy użyteczności, warto również dodać informacje na temat feedbacku od użytkowników oraz napotkanych trudności.

Niezbędne są też wizualizacje: wykresy, diagramy czy tabele, które pomogą czytelnikowi zrozumieć i zinterpretować uzyskane dane. Po prezentacji wyników ważna jest ich głęboka analiza, interpretacja w kontekście wcześniej postawionych celów oraz omówienie ewentualnych odchyleń od oczekiwań. Pamiętaj jednak, że sam wykres to za mało. Należy wskazać czytelnikowi, które aspekty uzyskanych wyników potwierdzają tezy pracy.

\section{Analiza aspektów prawnych}

W kontekście pracy inżynierskiej, analiza aspektów prawnych odgrywa kluczową rolę. Oprogramowanie, protokoły, algorytmy i różne metody, które zostały opracowane, podlegają różnym normom prawnym, które muszą być rozważone, aby zapewnić legalność oraz etyczność opracowanego rozwiązania.

\subsection{Prawo autorskie}

Wszystkie skrypty, kody źródłowe oraz dokumentacje stworzone w ramach tej pracy są chronione prawem autorskim. Jest to kluczowe, aby zabezpieczyć prawa własności intelektualnej autora oraz zapewnić odpowiednie licencjonowanie. Należy zastanowić się nad odpowiednią licencją, która umożliwi innym korzystanie z opracowanego rozwiązania, jednocześnie chroniąc prawa autora.

\subsection{Ochrona danych}

Jeśli w pracy zostały wykorzystane jakiekolwiek dane, szczególnie te dotyczące osób fizycznych, istotne jest przestrzeganie przepisów związanych z ochroną danych osobowych. W kontekście europejskim, należy zwrócić uwagę na zgodność z Rozporządzeniem o Ochronie Danych Osobowych (RODO).

\subsection{Prawo patentowe}

Jeśli opracowane rozwiązania są nowatorskie i mają potencjał komercyjny, warto rozważyć możliwość ubiegania się o patent. Chociaż proces patentowania w obszarze informatyki może być skomplikowany, chroni on przed nieautoryzowanym wykorzystaniem wynalazku.

\subsection{Licencjonowanie oprogramowania}

W zależności od charakteru rozwiązania oraz planowanej dystrybucji, istotne jest wybór odpowiedniej licencji. Może to być licencja open-source, która pozwala na szerokie wykorzystanie kodu przez społeczność, lub licencja komercyjna, która ogranicza dostęp do kodu i jego wykorzystanie.

\subsection{Etyka w nauce}

Oprócz ścisłych przepisów prawnych, ważne jest również przestrzeganie ogólnych zasad etyki w nauce. Obejmuje to uczciwość, transparentność oraz odpowiedzialność za stworzone rozwiązania i ich ewentualne wpływy na społeczeństwo.

Wnioski z niniejszego rozdziału mają na celu podkreślenie znaczenia aspektów prawnych w~procesie tworzenia nowych rozwiązań w dziedzinie informatyki stosowanej. Prawo w tej dziedzinie jest skomplikowane i dynamicznie się zmienia, dlatego istotne jest ciągłe monitorowanie aktualnych przepisów i dostosowywanie do nich opracowywanych rozwiązań.
