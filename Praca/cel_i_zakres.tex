\section*{Cel i zakres pracy}
\addcontentsline{toc}{section}{Cel i zakres pracy}
\label{chap:cel}
Cel i zakres pracy to istotne elementy, które precyzują, co autor chce osiągnąć w swojej pracy dyplomowej. Rozpocznij od sformułowania ogólnego celu swojej pracy. Należy pamiętać, by treść przedstawiona w streszczeniu i podsumowaniu była zgodna z główną treścią pracy. Ważne jest dbanie o spójność pracy: temat, zawartość, wstęp i podsumowanie powinny być ze sobą zgodne. 
Formułując cel pracy dyplomowej, powinieneś odpowiedzieć na pytanie: ,,Co zamierzam osiągnąć dzięki tej pracy?''. 

Cel pracy powinien być konkretny, osiągalny i mierzalny. Warto byłoby, gdyby był on istotny z punktu widzenia nauki, technologii lub społeczności. Cel powinien być precyzyjny tak, aby po przeczytaniu go, czytelnik dokładnie wiedział, czego można się spodziewać po przeczytaniu Twojej pracy.
np:

\begin{itemize}
   \item \textbf{Źle:} ,,Zbadanie technologii blockchain''.
    \item \textbf{Dobrze:} ,,Analiza możliwości zastosowania technologii blockchain w systemach gospodarki komunalnej''.
\end{itemize}
Upewnij się, że cel, który wyznaczasz, jest realny do osiągnięcia w ramach Twojej pracy dyplomowej. Idealnie, gdy cel można zweryfikować i zmierzyć. Na przykład, jeśli mówisz o optymalizacji czegoś, określ, jak zamierzasz mierzyć tę optymalizację. Najlepiej jest wyszukać w literaturze (pozycji bibliograficznej) w której rozwiązywany jest podobny problem i zastosować podobną metodologię badawczą. Dobrze gdy czynniki są mierzalne i nie związane są bezpośrednio z interakcją czy opinią użytkowników, co sprawia, że są one bardziej obiektywne i nie zależą od subiektywnych odczuć osób testujących.
Mierzalne czynniki:
\begin{enumerate}
    \item \textbf{Wydajność systemu:} \\
    \textit{Metoda pomiaru:} Monitorowanie i porównywanie czasu odpowiedzi systemu oraz przepustowości przed i po wprowadzeniu optymalizacji.

    \item \textbf{Zużycie zasobów:} \\
    \textit{Metoda pomiaru:} Analiza zużycia pamięci RAM, CPU oraz dysku w określonych warunkach operacyjnych.

    \item \textbf{Stabilność systemu:} \\
    \textit{Metoda pomiaru:} Monitorowanie i analiza liczby awarii lub błędów systemu w określonym czasie.
    
    \item \textbf{Czas przetwarzania danych:} \\
    \textit{Metoda pomiaru:} Porównanie czasu potrzebnego na przetworzenie określonej ilości danych przed i po wprowadzeniu zmian.
    
    \item \textbf{Skalowalność systemu:} \\
    \textit{Metoda pomiaru:} Testy obciążenia, aby określić, jak system radzi sobie z rosnącym obciążeniem.
    
    \item \textbf{Efektywność algorytmów:} \\
    \textit{Metoda pomiaru:} Analiza czasu wykonania oraz złożoności obliczeniowej algorytmu dla określonych danych wejściowych.
    
    \item \textbf{Zużycie energii przez urządzenie lub system:} \\
    \textit{Metoda pomiaru:} Monitorowanie zużycia energii w określonych warunkach pracy.
    
    \item \textbf{Poziom hałasu generowany przez maszynę lub urządzenie:} \\
    \textit{Metoda pomiaru:} Używanie decybelomierza do mierzenia poziomu hałasu w określonych warunkach.
    
    \item \textbf{Jakość kodu:} \\
    \textit{Metoda pomiaru:} Zastosowanie narzędzi do analizy jakości kodu, takich jak analizatory statyczne, które oceniają zgodność z określonymi standardami kodowania.
    
    \item \textbf{Poziom zabezpieczeń systemu:} \\
    \textit{Metoda pomiaru:} Przeprowadzenie testów penetracyjnych i ocena ilości i rodzaju znalezionych podatności.
     \item \textbf{Czas renderowania klatki:} \\
    \textit{Metoda pomiaru:} Monitorowanie i porównywanie czasu potrzebnego na renderowanie pojedynczej klatki przed i po wprowadzeniu optymalizacji.

    \item \textbf{Liczba klatek na sekundę (FPS):} \\
    \textit{Metoda pomiaru:} Użycie narzędzi do monitorowania FPS w czasie rzeczywistym podczas działania gry lub programu graficznego.

    \item \textbf{Złożoność sceny:} \\
    \textit{Metoda pomiaru:} Analiza liczby trójkątów, wierzchołków i obiektów w scenie.
    
    \item \textbf{Czas kompilacji shaderów:} \\
    \textit{Metoda pomiaru:} Monitorowanie czasu potrzebnego na kompilację shaderów graficznych.
    
    \item \textbf{Zużycie pamięci GPU:} \\
    \textit{Metoda pomiaru:} Użycie narzędzi do monitorowania zużycia pamięci karty graficznej podczas działania aplikacji.
    
    \item \textbf{Czas ładowania tekstur:} \\
    \textit{Metoda pomiaru:} Porównanie czasu potrzebnego na wczytanie tekstur do pamięci przed i po optymalizacji.
    
    \item \textbf{Efektywność technik cieniowania:} \\
    \textit{Metoda pomiaru:} Analiza jakości i wydajności różnych technik cieniowania, takich jak mapy cieni czy techniki ray tracing.
    
    \item \textbf{Zużycie zasobów podczas symulacji fizyki:} \\
    \textit{Metoda pomiaru:} Monitorowanie i analiza zużycia CPU i pamięci podczas symulacji fizycznych w grze.
    
    \item \textbf{Optymalizacja przepustowości:} \\
    \textit{Metoda pomiaru:} Analiza przepustowości i opóźnień w komunikacji między CPU a GPU.
    
    \item \textbf{Czas generowania światła globalnego:} \\
    \textit{Metoda pomiaru:} Porównanie czasu potrzebnego na obliczenie światła globalnego w~scenach o różnej złożoności.
  \item \textbf{Radiometric Consistency:} \\
    \textit{Opis:} Ocena, czy różne elementy w scenie są spójne pod względem oświetlenia i odbicia światła.

    \item \textbf{Geometric Consistency:} \\
    \textit{Opis:} Analiza spójności kształtów, krawędzi i struktury obiektów w scenie.

    \item \textbf{Temporal Consistency:} \\
    \textit{Opis:} W przypadku animacji lub filmów, ocena, czy renderowanie jest spójne w czasie, czyli czy nie ma niespójności w realizmie między klatkami.

    \item \textbf{Spectral Consistency:} \\
    \textit{Opis:} Ocenia, czy kolory w renderowanej scenie są spójne z rzeczywistymi kolorami, które powinny być reprezentowane w danej scenie.

    \item \textbf{Depth Cue Consistency:} \\
    \textit{Opis:} Analiza głębi obrazu, takich jak rozmycie, perspektywa i okluzja, aby ocenić, czy są one spójne z rzeczywistym światem.

    \item \textbf{Texture and Material Consistency:} \\
    \textit{Opis:} Ocenia, czy tekstury i materiały używane w scenie są reprezentowane w sposób realistyczny i spójny.

    \item \textbf{Shadow and Reflection Consistency:} \\
    \textit{Opis:} Analiza cieni i odbić w renderowanej scenie pod kątem ich spójności i realizmu.

    \item \textbf{Ambient Occlusion Consistency:} \\
    \textit{Opis:} Ocena, czy obszary w cieniu są odpowiednio ciemne i czy obszary w świetle są odpowiednio jasne.
   \item \textbf{Testy jednostkowe:} \\
    \textit{Opis:} Są to testy, które sprawdzają poszczególne jednostki kodu, takie jak funkcje lub metody, pod kątem poprawności działania.
    \textit{Przykład:} ,,Wszystkie funkcje w module XYZ muszą przejść testy jednostkowe bez błędów''.

    \item \textbf{Testy integracyjne:} \\
    \textit{Opis:} Testy te oceniają interakcje między różnymi modułami lub komponentami oprogramowania.
    \textit{Przykład:} ,,System musi poprawnie integrować się z bazą danych i serwisami zewnętrznymi''.

    \item \textbf{Testy wydajnościowe:} \\
    \textit{Opis:} Ocena, jak dobrze oprogramowanie działa pod obciążeniem lub w określonych warunkach.
    \textit{Przykład:} ,,Aplikacja musi obsługiwać co najmniej 10 000 użytkowników jednocześnie bez znaczącego spadku wydajności''.

    \item \textbf{Analiza kodu:} \\
    \textit{Opis:} Używanie narzędzi do analizy statycznej kodu w celu identyfikacji potencjalnych problemów lub niezgodności z standardami.
    \textit{Przykład:} ,,Kod musi być wolny od krytycznych błędów według narzędzia analizy kodu A''.

    \item \textbf{Metryki kodu:} \\
    \textit{Opis:} Określone metryki, takie jak złożoność cyklomatyczna, długość metody czy pokrycie kodu testami, które muszą spełniać określone kryteria.
    \textit{Przykład:} ,,Złożoność cyklomatyczna każdej funkcji nie może przekraczać 10''.

\end{enumerate}

Dopuszczalne (a nawet wskazane) są parametry złożone np. kwadratowy wskaźnik jakości: ${\bf x}^T {\bf A x}$, gdzie $\bf x$ to wektor wyżej wymienionych cech, a $\bf A$ to dodatnio określona macierz. Innym przykładem jest zdefiniowanie frontu Pareto.
W informatyce, testy bez udziału użytkowników są cenne, a bezpośrednie badania z udziałem użytkowników, takie jak metoda ,,Think Aloud'', mogą dostarczyć głębszych i bardziej szczegółowych informacji na temat interakcji użytkownika z oprogramowaniem. Kluczem jest jednak staranne planowanie i stosowanie odpowiedniej metodologii badawczej. Zwłaszcza w kontekście gier i grafiki komputerowej, bezpośrednie testy z~udziałem użytkowników mogą dostarczyć cennych informacji na temat użyteczności, satysfakcji użytkownika i innych aspektów interaktywnego oprogramowania. 
\begin{enumerate}[label=\arabic*.]
    \item \textbf{Think Aloud (Myślenie na głos):}
    \begin{itemize}
        \item Użytkownicy są proszeni o wykonywanie określonych zadań z oprogramowaniem, jednocześnie opisując swoje myśli i uczucia na głos.
    \end{itemize}

    \item \textbf{Cognitive Walkthrough (Kognitywne przechodzenie):}
    \begin{itemize}
        \item Eksperci oceniają łatwość nauki systemu przez nowych użytkowników, krok po kroku analizując proces wykonywania określonych zadań.
    \end{itemize}

    \item \textbf{Heuristic Evaluation (Ewaluacja heurystyczna):}
    \begin{itemize}
        \item Eksperci oceniają interfejs użytkownika na podstawie predefiniowanych zestawów ,,heurystyk'' lub zasad dobrej praktyki.
    \end{itemize}

    \item \textbf{Contextual Inquiry (Badanie kontekstowe):}
    \begin{itemize}
        \item Badacze obserwują użytkowników w ich naturalnym środowisku, aby zrozumieć, jak używają produktu w rzeczywistych warunkach.
    \end{itemize}

    \item \textbf{Remote Usability Testing (Zdalne testy użyteczności):}
    \begin{itemize}
        \item Użytkownicy testują produkt w swoim własnym środowisku, podczas gdy badacze obserwują i rejestrują ich działania zdalnie.
    \end{itemize}

    \item \textbf{Eye Tracking (Śledzenie ruchów oczu):}
    \begin{itemize}
        \item Za pomocą specjalnych urządzeń badacze analizują, na które elementy interfejsu użytkownicy najczęściej patrzą i jak przemieszczają się ich oczy po ekranie.
    \end{itemize}

    \item \textbf{A/B Testing (Testy A/B):}
    \begin{itemize}
        \item Dwie różne wersje interfejsu są prezentowane różnym grupom użytkowników, aby zobaczyć, która wersja jest bardziej skuteczna w osiągnięciu określonych celów.
    \end{itemize}

    \item \textbf{Surveys and Questionnaires (Ankiety i kwestionariusze):}
    \begin{itemize}
        \item Narzędzia służące do zbierania opinii i opinii od dużej liczby użytkowników na temat produktu lub interfejsu.
    \end{itemize}
\end{enumerate}

Zdefiniowanie celu pracy pozwoli nam na czytelne określenie kamieni milowych. Powinno to zostać skonsultowane z promotorem.

Zakres pracy określa, co dokładnie zostanie omówione w Twojej pracy, jakie metody zostaną użyte i co zostanie wykluczone. Uzasadnienie wyboru tematu pozwala czytelnikowi zrozumieć, dlaczego dany temat jest ważny i dlaczego warto się nim zająć.
Określ, jakie aspekty tematu zostaną omówione. Na przykład, jeśli Twoja praca dotyczy technologii blockchain w systemach gospodarki komunalnej, zakres może obejmować rodzaje technologii blockchain, potencjalne zastosowania w gospodarce komunalnej oraz analizę kosztów wprowadzenia takiego rozwiązania.  Jest równie ważne, aby określić, czego w pracy nie będzie. Może to obejmować pewne technologie, które są mniej istotne dla Twojego tematu, lub aspekty, które są poza zakresem Twojej analizy. Wyjaśnij, dlaczego wybrany przez Ciebie temat jest ważny w chwili pisania pracy. Może to być związane z aktualnymi trendami w branży, nowymi odkryciami naukowymi lub społeczną potrzebą.Możesz również wspomnieć, dlaczego ten temat jest dla Ciebie ważny lub interesujący. Jeśli zauważyłeś, że jest niewiele badań na dany temat, a uważasz, że jest on istotny, to jest to doskonałe uzasadnienie dla wyboru Twojego tematu.

Przykład:
\begin{quote}
Celem niniejszej pracy jest zbadanie i weryfikacja potencjału technologii blockchain w kontekście optymalizacji baz danych w systemach informatycznych gospodarki komunalnej. Centralnym punktem zaproponowanej analizy jest badanie efektywności i szybkości przetwarzania transakcji w środowiskach baz danych zintegrowanych z blockchainem w porównaniu do tradycyjnych baz danych. W ramach badania przeprowadzony zostanie benchmarking, który będzie oceniał wydajność operacji CRUD (tworzenie, odczyt, aktualizacja, usuwanie) w różnych warunkach obciążeniowych. Zakres pracy obejmuje także analizę poziomu bezpieczeństwa, stabilności oraz skalowalności rozwiązań bazodanowych opartych na blockchainie w stosunku do standardowych systemów bazodanowych. Z uwagi na techniczne aspekty tematu, omówione zostaną również główne algorytmy konsensusu w blockchainie oraz ich wpływ na wydajność i bezpieczeństwo systemów. Motywacją do wyboru tego zagadnienia jest rosnąca popularność technologii blockchain w dziedzinie informatyki stosowanej oraz przekonanie o jej potencjalnym wpływie na przyszłość systemów baz danych w sektorze usług komunalnych.
\end{quote}

