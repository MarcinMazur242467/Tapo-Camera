
\section*{Wnioski}
\addcontentsline{toc}{section}{Wnioski}
Na zakończenie rozdziału, w części poświęconej wnioskom, należy podsumować najważniejsze ustalenia i, jeśli to możliwe, zaproponować kierunki dalszej pracy, optymalizacji lub poprawek. Jeśli istnieją inne, podobne rozwiązania, warto także dodać porównanie uzyskanych wyników z tymi dostępnymi na rynku lub opisanymi w literaturze. Ostatecznie, najważniejsze jest, aby prezentacja wyników była przejrzysta, czytelna i konkretna.

\section*{Podsumowanie}
\addcontentsline{toc}{section}{Podsumowanie}
Rozdział ten stanowi kluczowy punkt pracy inżynierskiej, w którym wszystkie istotne elementy projektu i analizy zostają skumulowane i przedstawione w sposób zintegrowany. Jest to miejsce, gdzie podkreślana jest praktyczna wartość stworzonego rozwiązania technicznego oraz to, w~jaki sposób odpowiada ono na postawione wcześniej problemy inżynierskie.
W pierwszej kolejności, autor powinien podkreślić, czy cele pracy zostały osiągnięte i w jakim stopniu rozwiązanie techniczne spełnia założone wymagania.
Następnie ważne jest przedstawienie głównych osiągnięć projektowych, skupiając się na tym, co nowatorskiego wnosi prezentowane rozwiązanie i jakie korzyści może przynieść w praktyce.
Autor powinien również odnieść się do ewentualnych trudności napotkanych w trakcie realizacji projektu, omawiając wyzwania i~sposoby ich pokonywania.
W kontekście inżynierskim, warto porównać opracowane rozwiązanie z~dostępnymi na rynku alternatywami, wskazując jego mocne strony oraz potencjalne obszary do dalszej optymalizacji.
Praktyczne zastosowanie stworzonego rozwiązania powinno być podkreślone, omawiając potencjalne korzyści dla firm, instytucji lub użytkowników końcowych.
Zakończenie powinno wskazywać na potencjalne kierunki dalszych prac nad rozwiązaniem, co można jeszcze udoskonalić, jakie funkcje dodać czy w jakich obszarach zastosować wynikowe rozwiązanie inżynierskie.
