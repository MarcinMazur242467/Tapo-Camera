\section{Metodologia i implementacja rozwiązania}
\label{sec:metodologia}

\subsection{Metoda Badawcza}
\label{subsec:metoda_badawcza}
\subsubsection{Double Diamond}
Model ten umożliwia skuteczną transformację problemu (zamknięty ekosystem Tapo) w konkretne rozwiązanie inżynierskie.
\subsubsection{Projektowanie i implementacja}

\subsection{Założenia}
\label{subsec:zalozenia}
Główne założenia dotyczą pracy w sieci lokalnej (LAN), minimalizacji opóźnień oraz 100\% wykorzystania narzędzi Open Source.

\subsection{Zastosowane narzędzia i technologie}
\label{subsec:narzedzia}
\subsubsection{Język Programowania: Python 3.13}
\subsubsection{Konteneryzacja: Docker}
Zapewnia hermetyzację środowiska i reprodukowalność wdrożenia na różnych platformach sprzętowych.
\subsubsection{Package Manager: uv}
\subsubsection{Audio, Video i Streaming}
\begin{itemize}
    \item ffmpeg
    \item OpenCV
    \item MoviePy
\end{itemize}
\subsubsection{Web Server}
\begin{itemize}
    \item Flask
    \item WebSocket's
\end{itemize}
\subsubsection{Kontrola Kamery: PyTapo}

\subsection{Architektura rozwiązania}
\label{subsec:architektura}
Przedstawienie architektury opartej na trzech warstwach: dostępu do kamery, logiki biznesowej (Flask/SocketIO) i warstwie prezentacji.

\subsection{Proces implementacji}
\label{subsec:proces_implementacji}
\subsubsection{Serwer http}
\subsubsection{Implementacja połączenia z kamerą}
\subsubsection{Client}
\subsubsection{API}
\subsubsection{Przechwytywanie audio}
\subsubsection{Przechwytywanie wideo}
\subsubsection{Sterowanie kamerą - PTZ}
\subsubsection{Algorytm wykrywania ruchu}
\subsubsection{Nagrywanie}
\subsubsection{Zapis}

\subsection{Podsumowanie}
\label{subsec:podsumowanie_roz3}