\section{Metodologia i implementacja projektu}
\label{chap:drugi}
Rozdział ten skupia się na przedstawieniu metodyki badawczej oraz procesu wdrożeniowego wykorzystanego w projekcie inżynierskim. W następnych sekcjach są omawiane narzędzia, technologie oraz kroki podjęte w celu osiągnięcia założonych celów. Opis ten umożliwia czytelnikowi zrozumienie sposobu realizacji projektu inżynierskiego.

\subsection{Wybór i analiza metodologii}
W tym podrozdziale opisano dokładnie wybraną metodykę, dzięki której przeprowadzono badania oraz realizowano projekt. Wprowadzenie do metodyki pozwala zrozumieć, dlaczego została ona wybrana spośród innych dostępnych opcji. Następnie przedstawiono poszczególne etapy lub fazy metodologii, opisując podejmowane kroki oraz narzędzia używane w trakcie realizacji. Zakończenie tej sekcji skupia się na opisie procesu zbierania danych oraz metod analizy.

\subsection{Zastosowane narzędzia i technologie}
Sekcja ta omawia narzędzia, frameworki oraz technologie używane podczas pracy nad projektem inżynierskim. Po krótkim wprowadzeniu, każde narzędzie i technologia jest opisana, podkreślając ich rola oraz znaczenie w kontekście projektu.

\subsection{Projekt i specyfikacja}
Tutaj prezentowane są kluczowe elementy projektu, takie jak schematy, diagramy czy interfejsy. Po wprowadzeniu do głównych założeń projektowych, omawiane są szczegóły architektury oraz techniczne aspekty projektu, takie jak protokoły komunikacyjne czy algorytmy. Część ta kończy się specyfikacją, która przedstawia szczegółowy opis modułów i funkcji.

\subsection{Etap implementacji}
Ten podrozdział skupia się na praktycznej stronie projektu, przedstawiając kroki podjęte w~celu wdrożenia założeń projektowych. Po wprowadzeniu, które tłumaczy cele tego etapu, autor przedstawia szczegóły implementacji, w tym decyzje podjęte w trakcie tego procesu. Kluczowe fragmenty kodu źródłowego są również prezentowane, aby dać czytelnikowi wgląd w techniczne aspekty projektu. Do każdego listingu, tak jak rysunku, czy tabeli powinno być odniesienie w~tekście (listing~\ref{lst:pythonhw}).

\begin{lstlisting}[caption={Przykładowy kod w języku Python}, label=lst:pythonhw]
def hello_world():
    print("Hello, world!")

hello_world()
\end{lstlisting}

\subsection{Podsumowanie}
W zakresie podsumowania implementacji, autor ma za zadanie przedstawić zbiorcze wnioski oraz rezultaty osiągnięte w procesie realizacji projektu lub badań. Jest to etap, w którym można skoncentrować się na podkreśleniu kluczowych osiągnięć oraz wyciągnięciu istotnych wniosków z przeprowadzonej implementacji.
