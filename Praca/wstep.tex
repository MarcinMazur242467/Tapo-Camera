\section*{Wstęp}
\addcontentsline{toc}{section}{Wstęp}
\label{chap:Wstęp}
\markboth{WSTĘP}{WSTĘP}
Wstęp w pracy inżynierskiej wprowadza czytelnika w kontekst tematu pracy, zapoznając go z ogólnym tłem, które otacza badany problem lub projekt. We wstępie powinno się znaleźć krótkie przedstawienie obszaru, w którym została przeprowadzona praca, wskazanie aktualności problemu oraz podkreślenie jego znaczenia. Uświadom czytelnikowi, dlaczego omawiane zagadnienie jest ważne w kontekście inżynierii i technologii. Możesz podać krótką historię badanego zagadnienia, główne osiągnięcia w tym obszarze oraz wspomnieć o ewentualnych lukach w wiedzy lub problemach, które nadal pozostają nierozwiązane. Uzasadnij, dlaczego zdecydowałeś się na ten konkretny temat.

Jakie wydarzenia, trendy lub potrzeby rynku skłoniły Cię do podjęcia się tego zagadnienia. Przedstaw w skrócie używane metody badawcze lub podejście do analizy. Możesz wspomnieć, jakie narzędzia, technologie czy metody zostały użyte w pracy, bez wchodzenia w szczegóły (te zostaną przedstawione w osobnym rozdziale). Opisz, jak zorganizowana jest Twoja praca. Przedstaw krótko tematykę każdego rozdziału, aby czytelnik wiedział, czego się spodziewać. Pamiętaj, że wstęp ma na celu zainteresowanie czytelnika i przygotowanie go do głównego tematu pracy. Ma budować kontekst i stawiać Twoje badanie na tle innych prac w tej dziedzinie. Warto też dodać, że niektóre prace inżynierskie mogą mieć specyficzne wymagania dotyczące struktury wstępu, więc warto dostosować się do wytycznych wydziału czy promotora. Pamiętaj że powinieneś we wstępie dobrze nastroić recenzenta ;). Wstęp najczęściej piszemy już po skończeniu wszystkich rozdziałów pracy, zwykle ma około strony, chociaż w żadnej mierze nie oznacza to, że tak być musi. Bardzo rozsądnie jest poprosić o przeczytanie \textbf{całej} pracy kogoś, kto nie jest głęboko zaangażowany w jej realizację, być może nie jest nawet informatykiem (mama, tata, siostra, brat, ...). Ktoś taki skupi się na tekście i wskaże fragmenty niejasne. Jeżeli przyjmujecie taką radę, w przypadku wątpliwości, proszę nie tłumaczyć czytelnikowi dlaczego nie ma racji, ale poprawić tekst tak, by sam zrozumiał intencje autora.