\section{Metodologia i implementacja rozwiązania}
\label{sec:metodologia}

Poprzednie rozdziały dokonały teoretycznej dekonstrukcji technologii kamer IP (Rozdział~1) oraz przeprowadziły szczegółową analizę studium przypadku — kamery TP-Link Tapo C200 (Rozdział~2). Analiza ta zidentyfikowała kluczowy problem badawczy: fundamentalny konflikt między potencjałem sprzętowym urządzenia a ograniczeniami narzuconymi przez zamknięty ekosystem producenta (tzw. \textbf{„vendor lock-in”}).

Niniejszy rozdział przechodzi od teorii do praktyki. Stanowi on techniczną odpowiedź na zdefiniowane wyzwania. Opisany zostanie kompletny proces projektowy i wdrożeniowy – od wybranej metodyki badawczej, przez architekturę systemu, aż po szczegóły implementacji poszczególnych komponentów. Celem jest budowa autorskiego, otwartego rozwiązania programistycznego, które uwalnia pełen potencjał kamery i realizuje cele postawione w niniejszej pracy.

\subsection{Metodyka Projektowa}
\label{subsec:metodyka_projektowa}

\subsubsection{Double Diamond}
\label{subsubsec:double_diamond}

Model Double Diamond (Podwójny Diament) jest ustrukturyzowaną metodyką procesową, pierwotnie sformalizowaną przez British Design Council w 2005 roku. Stanowi ona mapę procesu projektowego, którego celem jest efektywne nawigowanie od wstępnej idei do wdrożonego rozwiązania, przy jednoczesnym zarządzaniu złożonością i niepewnością.

Metodyka ta jest fundamentalna dla współczesnego projektowania (w tym inżynierii oprogramowania, projektowania produktów i usług) i bazuje na koncepcji myślenia projektowego (Design Thinking).

Nazwa modelu pochodzi od jego wizualnej reprezentacji jako dwóch sąsiadujących rombów („diamentów”). Każdy diament reprezentuje sekwencję dwóch typów myślenia:

\begin{itemize}
    \item \textbf{Myślenie Rozbieżne:} Faza otwierania „diamentu”. Polega na eksploracji, generowaniu dużej liczby pomysłów, zbieraniu szerokiego spektrum danych i powstrzymywaniu się od oceny. Celem jest poszerzenie perspektywy i zrozumienie kontekstu.
    
    \item \textbf{Myślenie Zbieżne:} Faza zamykania „diamentu”. Polega na syntezie, analizie, krytycznej ocenie i podejmowaniu decyzji. Celem jest zawężenie opcji i wyłonienie konkretnego kierunku działania.
\end{itemize}

Model zakłada, że aby opracować właściwe rozwiązanie, należy najpierw dogłębnie zrozumieć i zdefiniować właściwy problem.

\subsubsection*{Diament 1: Przestrzeń Problemu}
Celem tego etapu jest zidentyfikowanie i precyzyjne zdefiniowanie kluczowego problemu, który ma zostać rozwiązany.

\begin{enumerate}
    \item \textbf{Faza Odkrywania (Discover) – Dywergencja} Jest to faza intensywnych badań (research) i empatii. Zespół projektowy wychodzi poza własne założenia, aby zrozumieć rzeczywisty kontekst użytkownika i zidentyfikować jego niezaspokojone potrzeby.
    
    \item \textbf{Faza Definiowania (Define) – Konwergencja} W tej fazie następuje synteza danych zebranych podczas Odkrywania. Zespół filtruje i analizuje informacje, szukając wzorców i kluczowych wyzwań. Celem jest przekształcenie rozproszonych obserwacji w klarowną i mierzalną definicję problemu.
\end{enumerate}

\subsubsection*{Diament 2: Przestrzeń Rozwiązania}

\begin{enumerate}
    \setcounter{enumi}{2} % Kontynuacja numeracji od 3
    \item \textbf{Faza Rozwijania (Develop) – Dywergencja} Mając jasno zdefiniowany problem, zespół ponownie przechodzi w tryb dywergencyjny, aby wygenerować jak najszerszy wachlarz potencjalnych rozwiązań. Kładzie się nacisk na ilość, a nie jakość, oraz na kreatywność i multidyscyplinarność.
    
    \item \textbf{Faza Dostarczania (Deliver) – Konwergencja} Jest to ostatnia faza, skupiona na testowaniu, walidacji i iteracyjnym udoskonalaniu wybranych koncepcji. Rozwiązania są poddawane rygorystycznym testom z udziałem użytkowników, aby zidentyfikować błędy i obszary do poprawy, a następnie zawęzić wybór do jednego, optymalnego rozwiązania gotowego do wdrożenia.
\end{enumerate}

\subsubsection*{Zastosowanie modelu w niniejszej pracy}

\begin{enumerate}
    \item \textbf{Odkrywanie (Discover):} Tę fazę reprezentuje research przeprowadzony w Rozdziałach 1 i 2. Zbadano ogólne działanie kamer IP, a następnie przeanalizowano specyfikę Tapo C200, identyfikując jej otwarte porty (RTSP) oraz zamknięte, własnościowe API do sterowania PTZ.
    
    \item \textbf{Definiowanie (Define):} Zebrane informacje skumulowano do konkretnego problemu: \textbf{vendor lock-in} uniemożliwia lokalną kontrolę. Celem pracy stało się więc stworzenie lokalnego systemu dającego pełną kontrolę.
    
    \item \textbf{Rozwój (Develop):} W tej fazie nastąpił brainstorming nad architekturą rozwiązania. Rozważano różne technologie i narzędzia (np. gotowe platformy vs. własna aplikacja). Zdecydowano się na elastyczny stos technologiczny, który umożliwi realizację wszystkich celów.
    
    \item \textbf{Dostarczanie (Deliver):} Wybrano konkretne, optymalne rozwiązanie: aplikacja webowa oparta na Pythonie, Flasku i WebSockets, wykorzystująca bibliotekę PyTapo (do sterowania) oraz OpenCV i FFmpeg (do analizy wideo), całość hermetyzowana w Dockerze. Implementacja tego rozwiązania stanowi dalszą część niniejszego rozdziału.
\end{enumerate}




\subsection{Architektura rozwiązania}
\label{subsec:architektura}
Przedstawienie architektury opartej na trzech warstwach: dostępu do kamery, logiki biznesowej (Flask/SocketIO) i warstwie prezentacji.
\subsubsection*{Wzorzec Klien-Server}
\subsubsection*{Diagram komponentow }
\subsubsection*{Diagram klas}
\subsubsection*{Diagram komunikacji9}








\subsection{Zastosowane narzędzia i technologie}
\label{subsec:narzedzia}
\subsubsection{Język Programowania: Python 3.13}
\subsubsection{Konteneryzacja: Docker}
\subsubsection{Package Manager: uv}
\subsubsection{Audio, Video i Streaming}
\begin{itemize}
    \item ffmpeg
    \item OpenCV
    \item MoviePy
\end{itemize}
\subsubsection{Web Server}
\begin{itemize}
    \item Flask
    \item WebSocket's
\end{itemize}
\subsubsection{Kontrola Kamery: PyTapo}

\subsection{Proces implementacji}
\label{subsec:proces_implementacji}
\subsubsection{Serwer http}
\subsubsection{Implementacja połączenia z kamerą}
\subsubsection{Client}
\subsubsection{API}
\subsubsection{Przechwytywanie audio}
\subsubsection{Przechwytywanie wideo}
\subsubsection{Sterowanie kamerą - PTZ}
\subsubsection{Algorytm wykrywania ruchu}
\subsubsection{Nagrywanie}
\subsubsection{Zapis}

\subsection{Podsumowanie}
\label{subsec:podsumowanie_roz3}